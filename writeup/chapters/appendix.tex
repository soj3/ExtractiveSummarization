\documentclass[../writeup.tex]{subfiles}

\begin{document}
\appendix


\chapter{Full results}\label{chapter:results}

\begin{figure}[H]
    \centering
    \begin{subfigure}{0.5\linewidth}
        \begin{tabular}{lrrr}
            \hline
            Rouge Metric & f-score & p-score & r-score \\
            \hline
            rouge-1      & 0.354   & 0.303   & 0.467   \\
            rouge-2      & 0.156   & 0.133   & 0.206   \\
            rouge-l      & 0.342   & 0.299   & 0.428   \\
            \hline
        \end{tabular}
        \caption{Baseline summary on CNN Dailymail}
    \end{subfigure}%
    \begin{subfigure}{0.5\linewidth}
        \begin{tabular}{lrrr}
            \hline
            Rouge Metric & f-score & p-score & r-score \\
            \hline
            rouge-1      & 0.266   & 0.432   & 0.212   \\
            rouge-2      & 0.074   & 0.118   & 0.06    \\
            rouge-l      & 0.287   & 0.287   & 0.302   \\
            \hline
        \end{tabular}
        \caption{Baseline summary on MultiNews}
    \end{subfigure}
    \caption{Baseline summary results}
    \label{appendix:fig:results:baseline_full}
\end{figure}


\begin{figure}[H]
    \centering
    \begin{subfigure}{0.5\linewidth}
        \begin{tabular}{lrrr}
            \hline
            Rouge Metric & f-score & p-score & r-score \\
            \hline
            rouge-1      & 0.317   & 0.252   & 0.468   \\
            rouge-2      & 0.127   & 0.101   & 0.188   \\
            rouge-l      & 0.306   & 0.252   & 0.42    \\
            \hline
        \end{tabular}
        \caption{SumBasicExtended on CNN Dailymail}
    \end{subfigure}%
    \begin{subfigure}{0.5\linewidth}
        \begin{tabular}{lrrr}
            \hline
            Rouge Metric & f-score & p-score & r-score \\
            \hline
            rouge-1      & 0.339   & 0.275   & 0.475   \\
            rouge-2      & 0.107   & 0.087   & 0.15    \\
            rouge-l      & 0.278   & 0.234   & 0.36s   \\
            \hline
        \end{tabular}
        \caption{SumBasicExtended on MultiNews}
    \end{subfigure}
    \caption{SumBasicExtended results}
    \label{appendix:fig:results:sumbasic_extended_full}
\end{figure}

\begin{figure}[H]
    \centering
    \begin{subfigure}{0.5\linewidth}
        \begin{tabular}{lrrr}
            \hline
            Rouge Metric & f-score & p-score & r-score \\
            \hline
            rouge-1      & 0.306   & 0.277   & 0.381   \\
            rouge-2      & 0.116   & 0.104   & 0.145   \\
            rouge-l      & 0.303   & 0.295   & 0.339   \\
            \hline
        \end{tabular}
        \caption{Penalized LexRank on CNN Dailymail}
    \end{subfigure}%
    \begin{subfigure}{0.5\linewidth}
        \begin{tabular}{lrrr}
            \hline
            Rouge Metric & f-score & p-score & r-score \\
            \hline
            rouge-1      & 0.353   & 0.333   & 0.411   \\
            rouge-2      & 0.112   & 0.104   & 0.133   \\
            rouge-l      & 0.295   & 0.301   & 0.311   \\
            \hline
        \end{tabular}
        \caption{Penalized LexRank on MultiNews}
    \end{subfigure}
    \caption{Penalized LexRank results}
    \label{appendix:fig:results:lexrank_full}
\end{figure}

\begin{figure}[H]
    \centering
    \begin{subfigure}{0.5\linewidth}
        \begin{tabular}{lrrr}
            \hline
            Rouge Metric & f-score & p-score & r-score \\
            \hline
            rouge-1      & 0.303   & 0.252   & 0.420   \\
            rouge-2      & 0.118   & 0.099   & 0.164   \\
            rouge-l      & 0.301   & 0.269   & 0.371   \\
            \hline
        \end{tabular}
        \caption{Similar Filtering (LSA) on CNN Dailymail}
    \end{subfigure}%
    \begin{subfigure}{0.5\linewidth}
        \begin{tabular}{lrrr}
            \hline
            Rouge Metric & f-score & p-score & r-score \\
            \hline
            rouge-1      & 0.298   & 0.407   & 0.259   \\
            rouge-2      & 0.078   & 0.106   & 0.068   \\
            rouge-l      & 0.276   & 0.256   & 0.318   \\
            \hline
        \end{tabular}
        \caption{Similar Filtering (LSA) on MultiNews}
    \end{subfigure}
    \caption{Similar Filtering (LSA) results}
    \label{appendix:fig:results:lsa_full}
\end{figure}



\chapter{Generated Summaries}\label{chapter:summaries}

\section{CNN/Daily Mail Summaries}\label{summaries:sec:cnn}

\begin{figure}[H]
    \centering
    {\small
        Experts question if packed out planes are putting passengers at risk.
        U.S consumer advisory group says minimum space must be stipulated.
        Safety tests conducted on planes with more leg room than airlines offer.}
    \caption{Human summary of article in CNN/Daily Mail}
    \label{appendix:fig:summaries:human_cnn}
\end{figure}

\begin{figure}[H]
    \centering
    {\small Ever noticed how plane seats appear to be getting smaller and smaller? With increasing numbers of people taking to the skies, some experts are questioning if having such packed out planes is putting passengers at risk. This week, a U.S consumer advisory group set up by the Department of Transportation said at a public hearing that while the government is happy to set standards for animals flying on planes, it doesnt stipulate a minimum amount of space for humans.}
    \caption{SumBasicExtended summary of article in CNN/Daily Mail}
    \label{appendix:fig:summaries:sumbasic_cnn}
\end{figure}

\begin{figure}[H]
    \centering
    {\small But these tests are conducted using planes with 31 inches between each row of seats, a standard which on some airlines has decreased, reported the Detroit News. While most airlines stick to a pitch of 31 inches or above, some fall below this. Tests conducted by the FAA use planes with a 31 inch pitch, a standard which on some airlines has decreased.}
    \caption{Continuous LexRank summary of article in CNN/Daily Mail}
    \label{appendix:fig:summaries:lexrank_cnn}
\end{figure}

\begin{figure}[H]
    \centering
    {\small Tests conducted by the FAA use planes with a 31 inch pitch, a standard which on some airlines has decreased. But these tests are conducted using planes with 31 inches between each row of seats, a standard which on some airlines has decreased, reported the Detroit News. While United Airlines has 30 inches of space, Gulf Air economy seats have between 29 and 32 inches, Air Asia offers 29 inches and Spirit Airlines offers just 28 inches.}
    \caption{Similar Filtering (LSA) summary of article in CNN/Daily Mail}
    \label{appendix:fig:summaries:filter_cnn}
\end{figure}



\section{MultiNews Summaries}\label{summaries:sec:multi}

\begin{figure}[h]
    \centering
    {\small A spaceship arrives on a distant planet that looks like a perfect new home for humans in Alien: Covenant.
        But if you know anything about Alien movies, you'll know theres only terror in store.
        Heres what critics are saying about the latest installment of the franchise, with director Ridley Scott of 1979s Alien returning: The filmmakers have finally managed to``dig the series out of its hole ,"Todd McCarthy puts it at the Hollywood Reporter.
        He calls Alien: Covenant``the most satisfying entry in the six-films-and-counting franchise since the first two.''
        Beautiful and gripping, it``feels vital"and is``keen to keep us on our toes right up to the concluding scene, which leaves the audience with such a great reveal that it makes you want to see the next installment tomorrow . ''
        Peter Howell at the Toronto Star agrees this flick``ranks among the better chapters"of the franchise.
        Scott``breaks new ground even while revisiting old concepts"and``brings back the visceral panic that fans expect . ''
        Actors Katherine Waterston, Danny McBride, Billy Crudup, and Michael Fassbender—who delivers``a grand performance times two"as two separate robots—also deserve high praise, he writes.
        The inclusion of Fassbenders David, from 2012s Prometheus, was an incredibly smart move, writes Joe Morgenstern at the Wall Street Journal.
        But there was little else that impressed him. Theres just``nothing new"to this``gore fest ,"he writes.
        He acknowledges, however, that``many Alien fans will come looking for something old, and thats in bloodily abundant supply . ''
        Chris Klimek had his own issues with the film.
        For example,``a religious subtext is introduced and then immediately abandoned ,"he writes at NPR.
        But he, too, had to marvel at Fassbender, whose``existential rap session"provides the``freshest part of the movie . ''
        Then again, Fassbenders David``is the only character in whom Scott seems truly interested ,"he writes.}
    \caption{Human summary of article in MultiNews}
    \label{appendix:fig:summaries:human_multi}
\end{figure}

\begin{figure}[h]
    \centering
    {\small Before chewing over the more predictable parts of Ridley Scott ’ s “ Alien: Covenant, ” let ’ s salute a really smart thing that Mr. Scott and his writers, John Logan and Dante Harper, have done with the latest edition of the “ Alien ” saga.
        Alien: Covenant: Film Review Michael Fassbender, Katherine Waterston and Billy Crudup lead the ensemble of Ridley Scotts second installment in the Alien prequel series.
        After the Alien series looked as though it had hit the rocks creatively (not for the first time) with the last entry, Prometheus, five years ago, savvy old master Ridley Scott has resuscitated it, and then some, with Alien: Covenant, the most satisfying entry in the six-films-and-counting franchise since the first two.
        Gripping through its full two hours and spiked with some real surprises, this beautifully made sci-fi thriller will immeasurably boost fan interest in the run of prequels which Scott has recently said will consist of at least two more films until the action catches up to the 1979 original.
        None comes to mind (Steven Spielberg made his first Indiana Jones adventure, Raiders of the Lost Ark, in 1981, two years after Alien was released).
        Its a matter of record that Scott will turn 80 later this year, and Clint Eastwood will be 87 when he starts his new film; from the evidence on the screen , 80 may well be the new 50 where some top helmers are concerned, especially those who, like Scott and Eastwood, make a new film almost every year.
        No matter that these aliens have been around far longer than most of the viewers who will see this film opening weekend have been alive; this entry feels vital, freshly thought out and keen to keep us on our toes right up to the concluding scene, which leaves the audience with such a great reveal that it makes you want to see the next installment tomorrow.
        The elegantly spare opening, in which a “ synthetic, ” Walter (Michael Fassbender), engages his “ father ” (an uncredited Guy Pearce) in a pointedly philosophical conversation, simply and effectively frames the thrust of the films central interest in human lifes origins and its prospects for survival.
        It barely showed the infamous beasties, choosing instead to batter senses with windy dialogue and events that seemed to betray basic elements of this decades-long saga.
        Alien: Covenant brings back the visceral panic that fans expect from the franchise, along with the show-stopping title creatures: among them the face-hugger, the chest-burster and the canoe-headed xenomorph of artist H.R.
        But, no, Walter, who sports an American, not British, accent, is an updated version of that all-purpose butler, factotum and technical wizard — a far friendlier iteration of the know-it-all computer Hal in 2001: A Space Odyssey.
        The Covenant carries 2,000 passengers and 1,140 human embryos, all frozen in cryogenic deep sleep and with an emphasis on preserving couples and families as they make the years-long journey to a planet called Origae-6.
        This couples-only orientation lends a fresh feel to this group of space travelers, and definitely cranks up the emotional distress quotient as partners start splitting open and giving birth to the wrong kind of offspring.
        But as inviting as are the beautiful landscapes, mountains and lakes, theres trouble lurking in the magnificent flora and fauna and, given the particulars of this bloody franchise, it doesnt take long for humans to fall ill and start bursting with nasty and ferocious critters they never imagined could spring from their innards. }

    \caption{SumBasicExtended summary of article in MultiNews}
    \label{appendix:fig:summaries:sumbasic_multi}
\end{figure}


\begin{figure}[h]
    \centering
    {\small Before chewing over the more predictable parts of Ridley Scott ’ s “ Alien: Covenant, ” let ’ s salute a really smart thing that Mr. Scott and his writers, John Logan and Dante Harper, have done with the latest edition of the “ Alien ” saga. Alien: Covenant: Film Review Michael Fassbender, Katherine Waterston and Billy Crudup lead the ensemble of Ridley Scotts second installment in the Alien prequel series. “ Prometheus, ” a sci-fi adventure set in the “ Alien ” universe, was concerned with such cosmic questions as the origin of life. After the Alien series looked as though it had hit the rocks creatively (not for the first time) with the last entry, Prometheus, five years ago, savvy old master Ridley Scott has resuscitated it, and then some, with Alien: Covenant, the most satisfying entry in the six-films-and-counting franchise since the first two. Alien: Covenant What does she see? Is there a director who has ever been artistically committed to a franchise as long as Scott has to the Alien series? And as in 2001, Alien: Covenant involves a long outer space voyage during which the 2,000 human passengers, along with 1,140 embryos, will linger in a deep-freeze sleep for several years while the humanoid plays watchdog. It also needs good actors to put it across, and Alien: Covenant more than delivers on that score. ( Watch also for cameos from Guy Pearce and James Franco, as they help connect the dots between Prometheus and Alien: Covenant . ) Katherine Waterston treads cautiously in a scene from Alien: Covenant.}
    \caption{Continuous LexRank summary of article in MultiNews}
    \label{appendix:fig:summaries:lexrank_multi}
\end{figure}


\begin{figure}[h]
    \centering
    {\small Alien: Covenant: Film Review Michael Fassbender, Katherine Waterston and Billy Crudup lead the ensemble of Ridley Scotts second installment in the Alien prequel series. After the Alien series looked as though it had hit the rocks creatively (not for the first time) with the last entry, Prometheus, five years ago, savvy old master Ridley Scott has resuscitated it, and then some, with Alien: Covenant, the most satisfying entry in the six-films-and-counting franchise since the first two. And as in 2001, Alien: Covenant involves a long outer space voyage during which the 2,000 human passengers, along with 1,140 embryos, will linger in a deep-freeze sleep for several years while the humanoid plays watchdog. Scott and the writers have achieved an outstanding balance in Alien: Covenant among numerous different elements: Intelligent speculation and textbook sci-fi presumptions, startlingly inventive action and audience-pleasing old standbys, philosophical considerations and inescapable genre conventions, intense visual splendor and gore at its most grisly. Before chewing over the more predictable parts of Ridley Scott ’ s “ Alien: Covenant, ” let ’ s salute a really smart thing that Mr. Scott and his writers, John Logan and Dante Harper, have done with the latest edition of the “ Alien ” saga. Katherine Waterston treads cautiously in a scene from Alien: Covenant. Alien: Covenant brings back the visceral panic that fans expect from the franchise, along with the show-stopping title creatures: among them the face-hugger, the chest-burster and the canoe-headed xenomorph of artist H.R. It also needs good actors to put it across, and Alien: Covenant more than delivers on that score. ( Watch also for cameos from Guy Pearce and James Franco, as they help connect the dots between Prometheus and Alien: Covenant . ) Not so Alien: Covenant, which breaks new ground even while revisiting old concepts.}
    \caption{Similar Filtering (LSA) of article in MultiNews}
    \label{appendix:fig:summaries:filter_multi}
\end{figure}


\chapter{Attribution}
\section{Individual Work}
Each named author of this report contributed an individual chapter
detailing a particular area of summarization methods, and provided
an implementation and analysis of one or several algorithms in this area.
As noted, the first three topic chapters can be attributed to:
\begin{itemize}
    \item Ch.~\ref{chapter:kennan}: Kennan LeJeune
    \item Ch.~\ref{chapter:david}: David Blinoce
    \item Ch.~\ref{chapter:sam}: Sam Jenkins
\end{itemize}

\section{Group Work}
The introduction covered in Ch.~\ref{chapter:intro} and comparison and analysis covered in Ch.~\ref{chapter:group}
was completed as a fully collaborative effort with equal contribution
from all group members.
\end{document}