\documentclass[11pt]{report}
\usepackage{fullpage}
\usepackage[utf8]{inputenc}
\usepackage[T1]{fontenc}
% \usepackage{newtxtext}
\usepackage{microtype}
\usepackage{hyperref}
\usepackage{amssymb}
\usepackage{amsfonts}
\usepackage{float}
\usepackage{amsmath}
\usepackage{amsthm}
\usepackage{graphicx}
\usepackage{wrapfig}
\usepackage{lipsum}
\usepackage{titling}
\usepackage{authblk}
\usepackage{subfiles}
\usepackage{subcaption}
\usepackage{algorithm2e}
\usepackage{multirow}
\usepackage{soul}
\usepackage[table,xcdraw]{xcolor}
% use \say{stuff} for quotations
\usepackage{dirtytalk}
% setup automatic bibliography generation
\usepackage[backend=bibtex,style=ieee]{biblatex}

\newcommand\MyBox[2]{
  \fbox{\lower0.75cm
    \vbox to 1.7cm{\vfil
      \hbox to 1.7cm{\hfil\parbox{1.4cm}{#1\\#2}\hfil}
      \vfil}%
  }%
}
\bibliography{citations}

\graphicspath{{images/}}
\title{Extractive Text Summarization:\\A Technical Survey and Comparison}
\author[1]{Kennan LeJeune}
\author[1]{David Blincoe}
\author[1]{Sam Jenkins}
\affil[1]{\textit{Department of Computer and Data Sciences\\Case Western Reserve University}}
\date{December 12, 2020}

\begin{document}
\maketitle
\begin{abstract}
  Extractive summarization, unlike abstractive summarization, involves summarizing documents by reusing portions
  of the original documents to overcome some of the obstacles associated with the NLG pipeline in
  abstractive summary generation.

  In this survey, we implement and compare a variety of
  extractive summarization techniques \autocite{brief-summarization-survey, text-summarization-techniques} including frequency-based, graph-based,
  and latent semantic analysis approaches. We demonstrate that all approach areas are capable of achieving strong
  results on both single and multiple document summarization using the
  CNN Dailymail \autocite*[]{dataset-cnn} and MultiNews \autocite*[]{dataset-multinews}
  datasets respectively, and analyze
  cases where the approaches perform particularly well or poorly. Finally, we implement and test
  new extensions to each set of
  approaches and examine the effectiveness of the extension in the context of relevant literature.
\end{abstract}
\tableofcontents{}
\subfile{chapters/introduction.tex}
\subfile{chapters/kennan.tex}
\subfile{chapters/david.tex}
\subfile{chapters/sam.tex}
\subfile{chapters/group.tex}
\subfile{chapters/appendix.tex}
\newpage
\printbibliography{}
\end{document}